\section{User Needs Assessment}
A critical consequence of funding EPO as a full sub-system since the start of construction is the ability to perform a formal user needs assessment. This section will describe the findings from this research that informed the design of our program. 

During 2017, the EPO Team hired a company that specialized in user experience design services, Guidea, to lead the User Needs Assessment work.  They interviewed 60 project stakeholders and about 30 members (?) of the four specific identified EPO audiences. 

Describe tests that were used (ordering science topics, asking what stakeholders would like to see from EPO, asking individuals how they would like to engage with astronomy topics or data, etc). 

 Consistent themes emerge from this assessment: it is critical to produce well-defined learning outcomes for activities, provide curated access to data, and develop simple, intuitive web interfaces. There is a necessity for all activities to be presented through mobile-friendly interfaces, and additionally, that materials for non-specialists to be presented in a way that require no software installation or downloading of data (especially important for educators introducing new activities to their classrooms). 

This work resulted in the creation of twelve personas that guided the design process through the rest of construction. 

\subsection{Personas}
A persona is a composite sketch of a key segment of your audience constructed to help deliver content that will be most relevant and useful to that audience. It inspires design, provides a benchmark through which to evaluate existing plans, and helps keep focus on what is important to our audiences.  

\section{User Testing Strategy}
In development of the program, all components were shown to different groups of people to gain feedback from our desired users on the success of the product. This appendix outlines the motivation behind user testing as well as the processes employed. 